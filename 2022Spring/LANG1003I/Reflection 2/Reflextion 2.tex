\documentclass[12pt]{article}
\usepackage{amsfonts, amsmath, amsthm, amstext, amssymb}
\usepackage{mathtools}
\usepackage{nccmath}
\usepackage{graphicx}
\usepackage{hyperref}
\usepackage{blindtext}
\usepackage{wrapfig}

\marginparwidth 0pt
\oddsidemargin -1 truecm
\evensidemargin  0pt
\marginparsep 0pt
\topmargin -2.6truecm
\linespread{1}
\textheight 24.5truecm
\textwidth 18.4 truecm
\newenvironment{remark}{\noindent{\bf Remark }}{\vspace{0mm}}
\newenvironment{remarks}{\noindent{\bf Remarks }}{\vspace{0mm}}
\newenvironment{question}{\noindent{\bf Question }}{\vspace{0mm}}
\newenvironment{questions}{\noindent{\bf Questions }}{\vspace{0mm}}
\newenvironment{note}{\noindent{\bf Note }}{\vspace{0mm}}
\newenvironment{summary}{\noindent{\bf Summary }}{\vspace{0mm}}
\newenvironment{back}{\noindent{\bf Background}}{\vspace{0mm}}
\newenvironment{conclude}{\noindent{\bf Conclusion}}{\vspace{0mm}}
\newenvironment{concludes}{\noindent{\bf Conclusions}}{\vspace{0mm}}
\newenvironment{dill}{\noindent{\bf Description of Dill's model}}{\vspace{0mm}}
\newenvironment{maths}{\noindent{\bf Mathematics needed}}{\vspace{0mm}}
\newenvironment{object}{\noindent{\bf Objective}}{\vspace{0mm}}
\newenvironment{notes}{\noindent{\bf Notes }}{\vspace{0mm}}
\newenvironment{theorem}{\noindent{\bf Theorem }}{\vspace{0mm}}
\newenvironment{example}{\noindent{\bf Example }}{\vspace{0mm}}
\newenvironment{examples}{\noindent{\bf Examples }}{\vspace{0mm}}
\newenvironment{lemma}{\noindent{\bf Lemma }}{\vspace{0mm}}
\newenvironment{solution}{\noindent{\it Solution}}{\vspace{2mm}}
\newcommand{\QED}{\fbox{}}

\graphicspath{{converted_graphics/}}
\begin{document}
\baselineskip 18 pt
\begin{center}
{\bf \Large LANG1003I\\ Reflection - Writing task}
\end{center}
\vspace{0.3cm}


The topic of my multimodal group project is the genius hour. Multimodality was used to enhance this project by bringing attention to important content and making it more interesting.



Fly-in images related to the important information are used in the video. For instance, the logos of Sha Tin College, 3M, and Google will pop up when it is mentioned as they are the important examples of how genius hour can be implemented in the local school and company. We decided to use this technique and minimize the words in the video as we thought this could attract our audience’s attention. Making the highlights of the presentation more memorable. Furthermore, not all students had a basic understanding of "genius hour" so the wordy video would be difficult for those individuals to understand. With the pictures, they could make a connection to our content and increase their ability to comprehend our messages.



Moreover, some upbeat background music was added to make the video's theme more relaxing and fun. As more people enjoy entertaining videos than academic videos, the audience would be more interested in our presentation. Also, a happy tone is used in the voice-over tracks of the video to create an enjoyable atmosphere. No one likes a flat-toned voice, so we made sure each other group members adopted a suitable intonation for different contexts in the video. For the wrap-up part, we tried to use a formal tone to do a summary because the content of the wrap-up is academic-oriented and we want to treat it seriously.



To conclude, many multimodalities including fly-in images, background music and intonation are used throughout our video presentation. These tools assisted us in enhancing both the quality and the meaning of the video by making it more stimulating and intelligible. Multimodal communication is fantastic for conveying information.






\end{document}