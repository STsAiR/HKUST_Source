\documentclass[12pt]{article}
\usepackage{amsfonts, amsmath, amsthm, amstext, amssymb}
\usepackage{mathtools}
\usepackage{nccmath}


\marginparwidth 0pt
\oddsidemargin -1 truecm
\evensidemargin  0pt
\marginparsep 0pt
\topmargin -2.6truecm
\linespread{1}
\textheight 24.5truecm
\textwidth 18.4 truecm
\newenvironment{remark}{\noindent{\bf Remark }}{\vspace{0mm}}
\newenvironment{remarks}{\noindent{\bf Remarks }}{\vspace{0mm}}
\newenvironment{question}{\noindent{\bf Question }}{\vspace{0mm}}
\newenvironment{questions}{\noindent{\bf Questions }}{\vspace{0mm}}
\newenvironment{note}{\noindent{\bf Note }}{\vspace{0mm}}
\newenvironment{summary}{\noindent{\bf Summary }}{\vspace{0mm}}
\newenvironment{back}{\noindent{\bf Background}}{\vspace{0mm}}
\newenvironment{conclude}{\noindent{\bf Conclusion}}{\vspace{0mm}}
\newenvironment{concludes}{\noindent{\bf Conclusions}}{\vspace{0mm}}
\newenvironment{dill}{\noindent{\bf Description of Dill's model}}{\vspace{0mm}}
\newenvironment{maths}{\noindent{\bf Mathematics needed}}{\vspace{0mm}}
\newenvironment{object}{\noindent{\bf Objective}}{\vspace{0mm}}
\newenvironment{notes}{\noindent{\bf Notes }}{\vspace{0mm}}
\newenvironment{theorem}{\noindent{\bf Theorem }}{\vspace{0mm}}
\newenvironment{example}{\noindent{\bf Example }}{\vspace{0mm}}
\newenvironment{examples}{\noindent{\bf Examples }}{\vspace{0mm}}
\newenvironment{lemma}{\noindent{\bf Lemma }}{\vspace{0mm}}
\newenvironment{solution}{\noindent{\it Solution}}{\vspace{2mm}}
\newcommand{\QED}{\fbox{}}

\usepackage{graphicx}
\graphicspath{{converted_graphics/}}
\begin{document}
\baselineskip 18 pt
\begin{center}
{\bf \Large LANG1003I U2L1 Note\\Why You Are Already a Natural?}
\end{center}
\vspace{0.3cm}

\section{CUE}
\begin{enumerate}
    \item How you can change your brain and become a better learner to learn faster and more effectively?
    \item How can we make the tree grow?
    \item How can our synapses work well?
\end{enumerate}


\section{NOTES}
\begin{enumerate}
    \item Our brain keeps changing
    \item Our brain is a "tree-like" system that will build "branches" or "twigs" when we are learning, exploring, thinking, and experiencing 
    \item Where neurons connect is called a synapse
    \item Thr neurons will also connect to one another to form a neuron \textbf{dentrites}
    \item To strengthen our synapses, we need to control our emotions because when we are learning, feel-good chemicals cause us to feel pleasure
    \item When we are stressed or anxious, our brain will release some chemical to make us think harder
    \item Feeling calm will create the best condition for learning
\end{enumerate}
\section{SUMMARY}
Our brain works like a tree. When we are learning, we are building a tree in the nervous system. We need good emotions to enhance learning efficiency. Moreover, we need to avoid bad feelings such as stress and anxiety when we are learning as it will slow down the pace.

Our brain works like a tree. When we are learning, we are building a tree in the nervous system.  The neurons in the background will connect to one another to form a network of neurons. Emotion is one of the factors that determine whether we learn effectively or not. Good emotions can make our brain release some chemicals to enhance learning efficiency. Moreover, we need to avoid bad feelings such as stress and anxiety when we are learning. It is because the chemicals released from our brain when we have bad emotions will make us think harder.




\end{document}