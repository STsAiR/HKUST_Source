\documentclass[12pt]{article}
\usepackage{amsfonts, amsmath, amsthm, amstext, amssymb}
\usepackage{mathtools}
\usepackage{nccmath}
\usepackage{graphicx}
\usepackage{hyperref}
\usepackage{blindtext}
\usepackage{wrapfig}

\marginparwidth 0pt
\oddsidemargin -1 truecm
\evensidemargin  0pt
\marginparsep 0pt
\topmargin -2.6truecm
\linespread{1}
\textheight 24.5truecm
\textwidth 18.4 truecm
\newenvironment{remark}{\noindent{\bf Remark }}{\vspace{0mm}}
\newenvironment{remarks}{\noindent{\bf Remarks }}{\vspace{0mm}}
\newenvironment{question}{\noindent{\bf Question }}{\vspace{0mm}}
\newenvironment{questions}{\noindent{\bf Questions }}{\vspace{0mm}}
\newenvironment{note}{\noindent{\bf Note }}{\vspace{0mm}}
\newenvironment{summary}{\noindent{\bf Summary }}{\vspace{0mm}}
\newenvironment{back}{\noindent{\bf Background}}{\vspace{0mm}}
\newenvironment{conclude}{\noindent{\bf Conclusion}}{\vspace{0mm}}
\newenvironment{concludes}{\noindent{\bf Conclusions}}{\vspace{0mm}}
\newenvironment{dill}{\noindent{\bf Description of Dill's model}}{\vspace{0mm}}
\newenvironment{maths}{\noindent{\bf Mathematics needed}}{\vspace{0mm}}
\newenvironment{object}{\noindent{\bf Objective}}{\vspace{0mm}}
\newenvironment{notes}{\noindent{\bf Notes }}{\vspace{0mm}}
\newenvironment{theorem}{\noindent{\bf Theorem }}{\vspace{0mm}}
\newenvironment{example}{\noindent{\bf Example }}{\vspace{0mm}}
\newenvironment{examples}{\noindent{\bf Examples }}{\vspace{0mm}}
\newenvironment{lemma}{\noindent{\bf Lemma }}{\vspace{0mm}}
\newenvironment{solution}{\noindent{\it Solution}}{\vspace{2mm}}
\newcommand{\QED}{\fbox{}}

\graphicspath{{converted_graphics/}}
\begin{document}
\baselineskip 18 pt
\begin{center}
{\bf \Large LANG1003I Academic Blog 2\\What A Genius(Hour)!}
\end{center}
\vspace{0.3cm}
Written on March 27, 2022 by KOO Kin Nam\\


\begin{wrapfigure}{r}{2in}
\includegraphics[scale = 0.2]{Photo2}
\end{wrapfigure}


Genius hour is getting more and more popular these days in many places, not only in schools but also in companies such as Google. Genius hours means schools will use 20\% of lesson time In the following, we will compare two articles about Genius Hour and their advantages.

Firstly, in the article, “Genius Hour and the 6 Essentials of Personalized Education”, Nichole Carter shared her experience when she took charge of the Genius Hour section at school. After that, she comes up with 6 important points for both students and teachers during Genius Hour. For example, communication between students and teachers can learn more about their students deeply (Carter, 2014).

Secondly, in the other article, “3 Benefits of Establishing a "Genius Hour", Meghan Bogardus Cortez mentioned 3 benefits of using Genius hour, which are enhancing students' creativity and critical thinking, creating a new item using technology, and a new way of standard.

When it comes to the execution of Genius Hours, both sources agree that planning is the key step. Farber suggested that objectives should be made for the Genius Hour lesson so that students will have a clear goal when they are trying to explore their interests (Cortez, 2017). At the same time, According to Carter (2014), mentioned that although the pace of the lesson is flexible, it should be built with structure. It shows that without doubt, the main purpose of the Genius hours is to allow students to explore their interests, but planning is also important for projects, such as setting the among of material for the project, or a timeline for the research.


However, they share a different point of view on the benefits of Genius Hours. Genius Hour can gradually improve the relationship between students and teachers (Carter, 2014). It is not only providing a chance for students to work on a project based on their interests, but also gives an opportunity for teachers to talk to their students about their interests and increases their understanding of the children. While it can enhance students' abilities, such as creativity and critical thinking (Cortex, 2017). Students have a higher probability to access different technology during the project. They need to utilize different tools to finish the project. For instance, students studying at a school in Kentucky need to share a OneNote notebook with others to present the progress of the project.





\end{document}