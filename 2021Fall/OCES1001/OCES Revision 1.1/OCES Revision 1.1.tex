\documentclass{report}
\title{OCES Revision}
\author{STsAiR}
\date{\today}
\begin{document}
\maketitle
\tableofcontents


\chapter{Module 1 Physical Characteristics of the Ocean}


\section{The Ocean: What is it and why study it}
The Ocean and atmosphere redistribute the excessive heat rceived by the Earth's surface via ocean/ atmosphere circulations.


\section{Approaches to studying the Ocean}
Approaches:
\begin{enumerate}
    \item Observation: Research Vessel, CTD, Drifters (to meansure ocean currents), Argo Float (Temperature and Salinity) and Head-mounted Sensors.
    \item Lab Experiments: Water Tank
    \item Modelling: High-Performance Computers (HPC) (Cheaper)
\end{enumerate}


\section{Oceanic Flows}
\begin{enumerate}
    \item Wind-driven Circulation (Upper Ocean): 
    \item [$\bullet$]Process in which surface winds push ocean water.
    \item [$\bullet$]Wind direction in the equator is from East to West.
    \item [$\bullet$]Warm current flows poleward and cold current flows equatorward.
    \item Ocean Gyres (Mid-latitude): 
    \item [$\bullet$]Wind-dricen circulation that is blocked by continental land.
    \item Circumpolar Currents/ Antarctic Circumpolar Current (ACC) (Southern Ocean):
    \item [$\bullet$]Wind goes from West to East.
    \item [$\bullet$]Currents that will not blocked by continental land.
\end{enumerate}


\section{Temperature, Salinity and Stratification}
\begin{enumerate}
    \item Sea Surface Temperature (SST) decreases from low latitude.
    \item The saltiest regions are mainly in the subtropical areas.
    \item In polar regions there are also substantial salt fluxes due to sea ice freezing and melting.
    \item Salnity decreases from surface to 1500m $-$ 2000m then increases again below 1500m $-$  2000m
    \item The Ocean Density is determined by both temperater and salinity. (Bowl-shaped pattern)
    \item The density increases toward high latitude and deeper ocean.
\end{enumerate}


\section{Impacts of Stratification on Oceanic Flows}
\begin{enumerate}
    \item Ocean Stratification:
    \item [$\bullet$]Sea water mainly follows constant density surfaces, rather than constant depths.
    \item Ocean Subduction:
    \item [$\bullet$]Wind-driven Gyres actually run deeper by sliding down the bowl of ocean density.
\end{enumerate}


\section{Where and how water sinks from surface to bottom}
\begin{enumerate}
    \item Wind-driven Circulation (Upper Ocean):
    \item [$\bullet$]More horizontal
    \item [$\bullet$]Driven by winds
    \item [$\bullet$]Transports ward/ cold water toward colder/ warmer parts of the Earth
    \item Overturning Circulation (Surface and Deeper Ocean):
    \item [$\bullet$]More vertical
    \item [$\bullet$]Driven by air-sea heat exchange and by salt input via ice formation
    \item [$\bullet$]Transports water, heat, saltm carbon dioxide, nutrients to deep sea
    \item Dense water sink in Polar regions
\end{enumerate}


\section{Where and how water rises from bottom to surface}
Internal Waves can cause ``Dead Zone'' and it rise in the ocean interior via wave breaking. It drive the bottom water upwards across density surface.


\section{Mesoscale Eddies}
\begin{enumerate}
    \item Size: 100 $-$ 200km
    \item Velocity: 1 m/s or higher
    \item Duration: Several weeks to months
    \item It can trap cold/ warm water and bring heat from one place to another.
    \item It can be find everywhere in the ocean.
\end{enumerate}


\section{MC Question}
\begin{enumerate}
    \item Where does dense water sink in the global ocean?
    \begin{enumerate}
        \item Polar region
        \item Subtropical region 
        \item Termperate region 
        \item Tropical region
    \end{enumerate}
    \item Which of the following is/are NOT blocked by lands?
    \begin{enumerate}
        \item Ocean gyres and circumpolar currents
        \item Circumpolar currents 
        \item Ocean gyres 
        \item Ocean gyres or circumpolar currents, depending on the location. 
    \end{enumerate}
    \item Which of the following statements is NOT true?
    \begin{enumerate}
        \item Wind-driven circulation transports water to deep ocean along constant density surfaces
        \item Wind-driven circulation is driven by heat and salinity
        \item None of the other answers is correct
        \item Compared with wind-driven circulation, overturning circulation is more vertical
    \end{enumerate}
    \item Over most of the ocean, the surface salt fluxes are related to
    \begin{enumerate}
        \item   warm and cold currents.
        \item   wind direction. 
        \item   rain and evaporation. 
        \item   current direction. 
    \end{enumerate}
    \item People at sea sometimes found their vessels unmovable no matter how they fire the engines. This could be due to the presence of
    \begin{enumerate}
        \item   circumpolar currents. 
        \item   surface currents. 
        \item   mesoscale eddies. 
        \item   internal waves. 
    \end{enumerate}
    \item Which of the following statements is true?
    \begin{enumerate}
        \item Excessive heat received by Earth’s surface at the equator is re-distributed to other regions by the atmosphere only.
        \item Excessive heat received by Earth’s surface at the equator is re-distributed to other regions by the ocean and the atmosphere.
        \item Excessive heat received by Earth’s surface at the equator is re-distributed to other regions by the ocean only.
        \item The ocean and the atmosphere are not involved in the heat distribution across different regions of the Earth.
    \end{enumerate}
    \item In what way do Mesoscale Eddies impact the ocean temperature?

    \begin{enumerate}
        \item   Mesoscale Eddies transport warm water only from one place to another in the ocean.
        \item   Mesoscale Eddies transport cold water only from one place to another in the ocean. 
        \item   Mesoscale Eddies are not related to ocean temperature at all. 
        \item   Mesoscale Eddies transport cold or warm water from one place to another in the ocean.
    \end{enumerate}
    \item Sea Surface Temperature
    \begin{enumerate}
        \item   remains largely unknown due to the lack of study. 
        \item   remains the same across different latitudinal zones. 
        \item   decreases from equator to poles. 
        \item   decreases from poles to equator. 
    \end{enumerate}
    \item How do ocean gyres adjust heat distribution of the ocean?
    \begin{enumerate}
        \item   Ocean gyres are not related to heat distribution in the ocean. 
        \item   Through warm current only. 
        \item   Through warm and cold currents. 
        \item   Through cold current only. 
    \end{enumerate}
    \item In polar regions, the salt fluxes are related to
    \begin{enumerate}
        \item   air temperature. 
        \item   wind and current directions. 
        \item   cold current. 
        \item   sea ice freezing and melting. 
    \end{enumerate}
\end{enumerate}


\chapter{Module 2a Glimpse of Marine Life}


\section{The Phylogeny of Marine Life}
\begin{enumerate}
    \item Life originated from hot and chemically-rich early Ocean
    \item Life evolved from inorganic molecules to organic molecules to functional independent units
    \item The three-domain system of phylogeny:
    \begin{enumerate}
        \item Prokaryotes (No organelle)
        \begin{enumerate}
            \item[$\rightarrow $]Bacteria
            \item[$\rightarrow$]Archaea  
        \end{enumerate}
        \item Eukarya
        \begin{enumerate}
            \item[$\rightarrow $]With organelles with contain DNA
            \item[$\rightarrow$]Evolved from Prokaryotes  
        \end{enumerate}
    \end{enumerate}
\end{enumerate}


\section{Bacteria}
\begin{enumerate}
    \item Usually single-celled and lack a membrane-bound nucleus
    \item Uniquitous and some live in extreme environments
    \item Bacterium is one type of microorganism
    \item Shapes and Arrangement:
    \begin{enumerate}
        \item Cocci: spherical
        \item Bacilli: rods 
        \item Vibrios: resemble rods, comma-shaped 
    \end{enumerate}
    \item Common features:
    \begin{enumerate}
        \item Cell envelope:
        \item [$\bullet$]Plasma membrane
        \item [$\bullet$]Cell Wall
        \item [$\bullet$]Capsule
        \item Cytoplasm
        \item External structures
    \end{enumerate}
\end{enumerate}


\section{Archaea}
\begin{enumerate}
    \item Bacteria and Archaea cannot be distinguished by simple microscopy
    \item They can be distinguished by unique rRNA gene sequences
    \item Archaea are lack peptidoglycan in cell walls
    \item They have unique membrane lipids
    \item Some of them have unusual metabolic characteristics
    \item They have different distributions:
    \begin{enumerate}
        \item Bacteria more abundant at top of ocean
        \item Archaea more abundant at bottom of ocean
    \end{enumerate}
\end{enumerate}


\section{Phytoplankton}
\begin{enumerate}
    \item Two lifestyles of marine life:
    \begin{enumerate}
        \item Plakton:
        \item [$\bullet$]Wandering, drift or swim weakly, unable to move consistently, such as shrimp
        \item Nekton:
        \item [$\bullet$]pelagic organisms that actively swim, such as fish
    \end{enumerate}
    \item Phytoplankton:
    \item [$\bullet$]They can be both Eukaryotes and Prokaryotes
    \item [$\bullet$]They are autotropic plankton that generate glucose by photosynthese $-$ the primary producers.
    \item [$\bullet$]They gets carbon from carbon dioxide and energy from sunlight.
    \item Phytoplankton generate a large amount of atmospheric oxygen.
    \item Major types of Phytoplankton:
    \item [$\bullet$]Picoplankton
    \item [$\bullet$]Diatoms
    \item [$\bullet$]Dinoflagellates
    \item [$\bullet$]Coccolithophores
\end{enumerate}


\section{Seaweeds and Marine Plants}
\begin{enumerate}
    \item Both Seaweeds and Seagrasses are important primary producers and Eukaryotics
    \item And they provided food and shelter to marine animals
    \item Seaweeds:
    \item [$\bullet$]Also known as Marine algae/ Macroalgae
    \item [$\bullet$]Have blade, stipe, and holdfast
    \item [$\bullet$]Can be used in dairy products (such as carrageenan)
    \item [$\bullet$]Can be used in culture microorganisms (such as agar)
    \item Seagrasses: 
    \item [$\bullet$]They are true plants that grow entirely underwater
    \item [$\bullet$]Trhy have flower, rhizome, and roots
    \item [$\bullet$]Around 60 species
    \item [$\bullet$]Pollen and tiny seeds are carried by water current or the face of animals
    \item Kelp is a type of brown Seaweed, and the largest seaweed
    \item It can be found in temperature and polar location
\end{enumerate}


\section{Animals}
\begin{enumerate}
    \item Invertebrates:
    \begin{enumerate}
        \item Animals without a backbone
        \item Around 97\% of anumals are invertebrates
        \item They do not have nervous system and brain but can reproduce
        \item They can be found on ocean floor attached to solid rocks
        \item There are numerous tiny pores for filter feeding on plankton and organic matter and carries wastes away
    \end{enumerate}
    \item Vertebrates:
    \begin{enumerate}
        \item Animals with a backbone
        \item Fish:
        \item [$\bullet$]Largest group of vertebrates 
        \item [$\bullet$]Over than half of 32000 known species of fishes are marine
        \item Reptiles
        \item [$\bullet$]Air-breathing, they have lungs instead of gills
        \item [$\bullet$]They are ``Cold-blooded''
        \item [$\bullet$]Their body are covered with scales
        \item [$\bullet$]They may lay eggs on land
        \item Mammals:
        \item [$\bullet$]They are ``Warm-blooded''
        \item [$\bullet$]They are active predators at the top of the food chain
    \end{enumerate}
\end{enumerate}


\section{Viruses}
\begin{enumerate}
    \item Virus is a noncellular particular that must infext a host to reproduce
    \item They are everywhere and part of our daily lives
    \item Each species of virus infect a particular group of host species
    \item The viral capsid is composed of repeated protein subunits
    \item The capsid packages the viral genome and delivers it into the host cell
    \item Differrent virus make different capsid forms
    \item Most of they are formed with ``Head'' and ``neck''
    \item The biomass of viruses are relatively low, but the abundance of viruses are the highest
    \item Importance of virus:
    \item [$\bullet$]Limiting host population density
    \item [$\bullet$]Selecting for host diversity 
\end{enumerate}


\section{MC Question}
\begin{enumerate}
    \item Which of the following is NOT a characteristic of marine mammals?
    \begin{enumerate}
        \item   Give live birth. 
        \item   Breathe air. 
        \item   Cold-blooded. 
        \item   Produce milk for the young. 
    \end{enumerate}
    \item Which of the following statements is NOT true?
    \begin{enumerate}
        \item   None of the other answers is correct. 
        \item   Some seaweeds are commercially important. 
        \item   Seagrasses are true plants. 
        \item   Kelp forests are highly productive. 
    \end{enumerate}
    \item Which of the following is NOT a characteristic of marine reptiles?
    \begin{enumerate}
        \item   None of the other answers is correct. 
        \item   Body covered with scales. 
        \item   Breathe air. 
        \item   All species lay their eggs in water. 
    \end{enumerate}
    \item Invertebrates include the following, EXCEPT
    \begin{enumerate}
        \item reptiles.
        \item   sponges. 
        \item   octopuses. 
        \item   sea stars. 
    \end{enumerate}
    \item Kelps belong to the following, EXCEPT
    \begin{enumerate}
        \item   brown algae. 
        \item   flowering plants. 
        \item   none of the other answers is correct. 
        \item   eukaryotes. 
    \end{enumerate}
    \item Which of the following is NOT an ecological role of phytoplankton in the marine environment?
    \begin{enumerate}
        \item   Primary producers. 
        \item   Primary consumers. 
        \item   Food for primary consumers. 
        \item   None of the other answers is correct. 
    \end{enumerate}
    \item Viruses are important in all ecosystems because
    \begin{enumerate}
        \item   they can kill their hosts such as bacteria. 
        \item   they help some host species to dominate over others. 
        \item   some of them are primary producers. 
        \item   they help balance the diversity of host species. 
    \end{enumerate}
    \item Seagrasses are important in the marine ecosystem because of all of the following, EXCEPT
    \begin{enumerate}
        \item   Seagrasses produce oxygen and absorb carbon dioxide through photosynthesis.
        \item   Seagrasses provide shelter for many marine animals. 
        \item   None of the other answers is correct. 
        \item   Seagrasses provide food for many marine animals. 
    \end{enumerate}
    \item   Which of the following statements is NOT true? 
    \begin{enumerate}
        \item   None of the other answers is correct. 
        \item   There are only a few truly marine plant species. 
        \item   All phytoplankton are eukaryotes. 
        \item   Archaea are prokaryotes. 
    \end{enumerate}
    \item Viruses
    \begin{enumerate}
        \item   are low in abundance in open ocean. 
        \item   all of the other answers are correct. 
        \item    must infect a host cell to reproduce. 
        \item   are high in biomass in open ocean. 
    \end{enumerate}
\end{enumerate}


\chapter{Module 2b Glimpse of Marine Life and Interactions}


\section{The Marine Food Web}
\begin{enumerate}
    \item The Marine Food Chain:
    \item [$\bullet$]Primary Producers
    \item [$\rightarrow$]Using sunlight and CO$_{2}$ to rpoduce their own food
    \item Consumers: Consume other living organisms
    \begin{enumerate} 
        \item Primary Comsumers
        \item [$\rightarrow$]Feeding on primary producers, Herbiovres (plant-eating)
        \item Secondary Comsumers
        \item [$\rightarrow$]Feeding on primary consumers, Carnoviores (meat-eating) 
    \end{enumerate} 

    \item [$\bullet$]Decomposers
    \item [$\rightarrow$]Break down waste by feeding on dead tissues of plants and animals, such as bacteria, sea worms and sea slugs
    \item Energy lost: excretion and respiration
    \item On average, only 10$\%$ of the nutrients and energy is transferred fromt he trophic level to the next.
\end{enumerate}


\section{Modes of Nutrition in the Marine Environment}
\begin{enumerate}
    \item Autotropic $-$ Production only
    \item Heterotroph $-$ Consumption only
    \item Mixotroph $-$ Both production and consumption, such as bacteria and plankton 
\end{enumerate}


\section{Photosynthesis}
\begin{enumerate}
    \item Photoautotrophs: 
    \item [$\bullet$]Autotrophs use sunlight as energy source
    \item Chemoautotrophs:
    \item [$\bullet$]autrophs may use energy stored in chemical compounds for the synthesis of organic food
    \item Photosynthesis:
    \item [$\bullet$]The Chlorophyll receives sunlight and release chemical energy ATP to turn H$_{2}$O + CO$_{2}$ into O$_{2}$ and gluvose/ sugar
    \item  The remaining organics are accumulated in the
    photoautotrophs for production of biomass such as growth     
    \item Gain in biomass (growth) = Primary Production
    \item Rate of primary production = Primary Productivity 
    \item Data of primary productivity are important not
    only for scientific understanding of the ocean but
    also for management of fisheries resources.
    \item Primary production regulates the gaseous composition in the atmosphere
\end{enumerate}


\section{Absorption of Dissolved Organic Matter}
\begin{enumerate}
    \item Dissolved Organic Matter (DOM) is organic waste in the seawater. 
    \item Photoautotrophs are the most important producers of DOM 
    \item Microorganisms are the most important contributor of DOM
    \item Viral shunt accelerates the release of DOM and helps to balance the ecosystem (avoiding dominance
    by certain species)    
\end{enumerate}


\section{Detritus Feeding}
\begin{enumerate}
    \item All life in the ocean produce detritus, such as decaying waste or debris rich in nutrients from marine life
    \item Detritus are suspended in the seawater ot deposited on the ocean floor
    \item Detritus is an important food source for many different types of marine organisms, from heterotrophic microbes to invertebrates
    \item Detritus come from solid waste of marine animals or dead biomass of both marine plant and anumals
    \item DOM is dissolved while detritus exists in the form of particulates
    \item When the detritus im the seawater column is drawn to the ocean dloor by gravity, it formed Marine Snow and they are important food source for the deep sea community
    \item Heterotrophic bacteria attached to detritus 
    \item Types of detritus feeders:
    \item [$\rightarrow$]Invertebrates:
    \begin{enumerate}
        \item Suspension Feeders:
        \item [$\bullet$]Animals that feed on particles floating (suspended) in the water
        \item Filter Feeders:
        \item [$\bullet$]Suspension feeders that actively filter food particles
        \item Deposit Feeders:
        \item [$\bullet$]Animals that feed on organic matter that settle on the bottom
    \end{enumerate}
    \item [$\rightarrow$]Active feeding (consume marine snow):
    \begin{enumerate}
        \item fish
        \item zooplankton
    \end{enumerate}
\end{enumerate}


\section{Predation} 
\begin{enumerate}
    \item predator and prey interact and never-ending co-evolve 
    \item Predator is the source of mortality for prey while prey is a source of nutrient and energy for predator 
    \item Their population size are like sine and cosine curve
    \item Predation tactics:
    \item [$\bullet$]Putsuit hunting
    \item [$\bullet$]Ambush (Sit and Wait) (involving Camouflage, which is change of color)
    \item Prey's tactics:
    \item [$\bullet$]Behavior 
    \item [$\bullet$]Camouflage (Protective coloration)
\end{enumerate}


\section{Scavenging}
\begin{enumerate}
    \item Scavengers (clean-up crew) consume dead body parts / tissues that are much larger in scale and they do not consume feces 
    \item Scavengers are important to nutrient recycling, through breaking down large pieces of dead tissues into smaller fragments (into feces) for detritivores 
    \item The feces produced by scanvengers are further utilized by detritivores
    \item Nothing is wasted
\end{enumerate}


\section{Parasitism}
\begin{enumerate}
    \item Parasite and host undergo continuous co-evolution
    \item Very often a parasite also uses the host’s body as living habitat and site for reproduction
    \item A parasite foes not intend to kill the host bu the host may suffer from disease due to secondary infection by bacteria or viruses 
    \item The cost of defense and the loss of nutrients to parasite are needed to balance 
\end{enumerate}


\section{Proto-cooperation}
\begin{enumerate}
    \item Four stages of parasitism:
    \begin{enumerate}
        \item Parasitism
        \item [$\bullet$]One-sided benefit, parasite harms the host 
        \item Commensalism
        \item [$\bullet$]ONe-sided benefit without harming the host 
        \item Proto-cooperation
        \item [$\bullet$]Mutual benefits and non-obligatory
        \item Mutualism
        \item [$\bullet$]Mutual benefits and obligatory
    \end{enumerate}
    \item Example of Proto-cooperation: The Hermit Crab lose food to the Sea Anemone while the crab can gain protection
\end{enumerate}


\section{Mutualism}
\begin{enumerate}
    \item The two parties are unable to live apart
    \item Zooxanthellae provide organic food to corals while corals provided shelter for zooxanthellae
\end{enumerate}


\section{MC Question}
\begin{enumerate}
    \item Marine Snow is an important food source for the following, EXCEPT
    \begin{enumerate}
        \item   zooplankton. 
        \item   deep-sea communities. 
        \item   fishes. 
        \item   phytoplankton. 
    \end{enumerate}
    \item Heterotrophs include the following, EXCEPT
    \begin{enumerate}
        \item   Zooplankton. 
        \item   Bacteria 
        \item   None of the other answers is correct. 
        \item Phytoplankton.
    \end{enumerate}
    \item Which of the following statements is ture?
    \begin{enumerate}
        \item   none of the other answers is correct. 
        \item   In general, only 10$\%$ of biomass and energy is transferred from one trophic level to the one below.
        \item   In general, only 10$\%$ of biomass and energy is utilized by the organisms in each trophic level.
        \item   In general, only 10$\%$ of biomass and energy is transferred from one trophic level to the one above.
    \end{enumerate}
    \item Which of the following is the key consumer of Dissolved Organic Matter (DOM) in the ocean?
    \begin{enumerate}
        \item   Seagrasses. 
        \item   Invertebrates. 
        \item   Fishes. 
        \item   Microorganisms. 
    \end{enumerate}
    \item Which of the following associations involves one party noticeably taking excess resources from another party? 
    \begin{enumerate}
        \item   Parasitism. 
        \item   Proto-cooperation. 
        \item   Commensalism. 
        \item   Mutualism. 
    \end{enumerate}
    \item Detritus
    \begin{enumerate}
        \item   are part of the DOM in the marine environment. 
        \item   are important food source for many marine organisms. 
        \item   are a product of photosynthesis. 
        \item   all of the other answers are correct. 
    \end{enumerate}
    \item Filter feeders feed on
    \begin{enumerate}
        \item   detritus that are deposited on the seafloor. 
        \item   All of the other answers are correct. 
        \item   DOM
        \item   detritus that float in seawater. 
    \end{enumerate}
    \item Which of the following is NOT true about Predation?
    \begin{enumerate}
        \item   Predator and prey populations cycle through time. 
        \item   Predator and prey interact. 
        \item   Predator and prey co-evolve. 
        \item   None of the other answers is correct. 
    \end{enumerate}
    \item Zooplankton are
    \begin{enumerate}
        \item   primary consumers. 
        \item   top carnivores. 
        \item   primary producers. 
        \item   secondary consumers. 
    \end{enumerate}
    \item Seaweed, seagrasses and phytoplankton are
    \begin{enumerate}
        \item   heterotrophs. 
        \item   photoautotrophs. 
        \item   mixotrophs. 
        \item   chemoautotrophs. 
    \end{enumerate}
\end{enumerate}


\chapter{Module 3a Marine Ecosystems: Coastal Physical Processes}
\section{Characteristics of Coastal and Shelf Sea}
\begin{enumerate}
    \item Shallow Water Depth:
    \begin{enumerate}
        \item Wind effect:
        \item [$\bullet$]Wind is inversely proportional to wind
        \item Friction effect:
        \item [$\bullet$]Bottom friction can be readily felt by water through the water column
        \item Amplification of Tide and Tidal Current
    \end{enumerate}
    \item Presence of Coastline:
    \begin{enumerate}
        \item Bloack the water on landside, resulting convergence or divergence
        \item Sea-level increases/ decreases, and leads surge
    \end{enumerate}
    \item Terrestrial influences (plume):
    \item [$\bullet$]Fresh water discharge from land forms higher sea level
    \item [$\bullet$]Change buoyancy and direction of the current
    \item [$\bullet$]Bring right inorganic nutrients and organic matter to enhance biological productivity 
    \item Open ocean influences:
    \item [$\bullet$]Due to large-scale circulation, eddies and internal wave.
\end{enumerate}


\section{Ekman Transport}
\begin{enumerate}
    \item The combined effect of wind forcing and Earch rotation leads to Ekman Transport (Drift)
    \item It is the average current within the Ekman spiral with is $x$-axis = Ekman transport and $y$-axis = diretion of wind
    \item It forms sea-level difference
\end{enumerate}


\section{Coastal Upwelling}
\begin{enumerate}
    \item Wind can induce Coastal Upwelling
    \item When Ekman transport moves surface water away from the coast (land), surface waters are replaced by water that wells up from below 
    \item In the Northern Hemisphere, coastal upwelling can be caused by winds from the north blowing along the west coast of a continent 
    \item The water will be replaced by the upwelled cold, deep, nutrient-laden water
    \item Ekman transport $\Rightarrow $ Upwelling $\Rightarrow $ High in nutrients $\Rightarrow $ Rich biological productivity (More than 50$\%$ of the world's annual commercial fishing occurs in the upwelling zone) $\Rightarrow $ Bursts of phytoplankton $\Rightarrow $ Build up of zooplankton $\Rightarrow $ Landing of fish
\end{enumerate}


\section{Coastal Downwelling}
\begin{enumerate}
    \item Areas of downwelling are often low in nutrients and therefore relatively low in biological productivity 
    \item Cold coastal water is transported downslope during downwelling
    \item Coastal downwelling can be caused by winds from the south blowing and water move towards the coast, water piles up and sinks towards the bottom 
\end{enumerate}


\section{Estuary and Estuarine}
\begin{enumerate}
    \item The place wherer lighter freshwater and heavier seawater meet formed estuary 
    \item Density difference between river and sea water and easterly or westerly winds 
\end{enumerate}


\section{Circulation Classification of Estuaries}
\begin{enumerate}
    \item Since the seawater colume from coean is relaticely stable, the ratio or type of estuary is mainly determined by freshwater volume from the river
    \item Salt Wedge Estuary:
    \item [$\bullet$]river volume larger than tidal volume 
    \item Highly Stratified Estuary:
    \item [$\bullet$]River colume comparable to tidal colume but still larger than tidal volume 
    \item [$\bullet$]Create instability and internal waves
    \item [$\bullet$]A net upward transport of mass and salt 
    \item [$\bullet$]The dense seawater seldom reaches the upper estuary 
    \item Slightly Stratified Estuary:
    \item [$\bullet$]River volume small compred to tidal colume
    Saltwater and freshwater mix at all depths 
    \item [$\bullet$]Salinity is greatst at the mouth of the estuary, and decreasess upstream
    \item Vertically Mixed Estuary:
    \item [$\bullet$]River volume insignificant compared to tidal volume 
    \item [$\bullet$]Efficient turbulent mixing 
    \item [$\bullet$]No distinction between upper and lower layers 
\end{enumerate}


\section{Tides}
\begin{enumerate}
    \item It is caused by a combination of the gravitational force of the moon and sun and the motion of the Earth
    \item Two types of tides:
    \item [$\bullet$]Flood Tide (High Tide)
    \item [$\bullet$]Ebb Tide (Low Tide)
    \item Tide formed by the influence of moon and it is twien that of the sun's influence 
    \item Tidal Bulges form by the attraction of the Earth and the moon 
    \item Spring Tides are formed when Earth, sun and moon are aligned
    \item Neap Tides are formed when Earch, sun and moon are at an angle to each other 
\end{enumerate}


\section{Waves}
\begin{enumerate}
    \item Wave are created by energy passing through water, not water mass, transporting energy from one location to another across the ocean's surface.
    \item Capillary waves, wind waves and swells are near surface due to the wind effects on the air/ water interface
    \item Internal waves occur within the water not on the surface when vertical density variations are present
    \item Tsunamis are very long waves generated by seismic disturbances of the sea bottom or shore
    \item Gravity is the restoring force for most of the waves
\end{enumerate}


\section{MC Question}
\begin{enumerate}
    \item Which of the following statements is NOT true about coastal waters?
    \begin{enumerate}
        \item   They are transitional systems between the land and the open ocean. 
        \item   None of the other answers is correct. 
        \item   Wind effect is weakened in coastal waters. 
        \item   Coastal waters have relatively high biological productivity. 
    \end{enumerate}
    \item The freshwater discharge from land influences all of the following parameters of coastal waters, EXCEPT
    \begin{enumerate}
        \item   the number of tidal cycle (s) per day. 
        \item   salinity. 
        \item   biological productivity. 
        \item   sea level. 
    \end{enumerate}
    \item The presence of the coastline is important to
    \begin{enumerate}
        \item   all of the other answers are correct. 
        \item   coastal upwelling. 
        \item   the sea level. 
        \item   coastal downwelling. 
    \end{enumerate}
    \item Ekman transport
    \begin{enumerate}
        \item   is caused by the water movement between the upper layer and the bottom layer.
        \item   none of these answers is correct. 
        \item   turns the ocean current at 90° to the left of the wind direction. 
        \item   affects sea levels. 
    \end{enumerate}
    \item Which of the following is TRUE about coastal upwelling?
    \begin{enumerate}
        \item   Coastal upwelling results in an even distribution of biologically productive zones across global oceans.
        \item   All of the other answers are correct. 
        \item   Warm and nutrient-rich water in the ocean surface is brought offshore, which enhances the biological productivity in the open ocean.
        \item   Upwelling can be initiated by wind. 
    \end{enumerate}
    \item The formation of coastal upwelling and downwelling is related  to the following,  EXCEPT
    \begin{enumerate}
        \item   Wind. 
        \item   Ekman transport. 
        \item   None of the other answers is correct. 
        \item   Friction effect. 
    \end{enumerate}
    \item Coastal waters next to estuaries have relatively high biological productivity because of
    \begin{enumerate}
        \item   terrestrial influences. 
        \item   strong coastal upwelling. 
        \item   its connection with the open ocean. 
        \item   strong coastal downwelling. 
    \end{enumerate}
    \item Estuarine circulation
    \begin{enumerate}
        \item   is formed because of the salinity difference between freshwater and seawater inputs.
        \item   is formed by river plume. 
        \item   is formed by intensified wind forcing in the estuary. 
        \item   enhances the mixing of surface water and deeper water in the ocean. 
    \end{enumerate}
    \item Which of the following is/are the determining factor (s) of estuary classification?
    \begin{enumerate}
        \item   River volume. 
        \item   River volume, seawater volume and the number of tidal cycle (s) per day.
        \item   Both river volume and seawater volume. 
        \item   Seawater volume. 
    \end{enumerate}
    \item Which of the following statements is true?
    \begin{enumerate}
        \item   Waves are merely formed by the Earth’s rotation and the gravitational force of the moon and sun.
        \item   Tides are not caused by external force and restoring force. 
        \item   Waves amplify as they reach the coast. 
        \item   Waves speed up as they reach the coast. 
    \end{enumerate}
\end{enumerate}


\chapter{Module 3b Marine Ecosystems: Subtidal, Intertidal and Estuaries}
\section{The Subtidal Ecosystems: Unvegetated Areas}
\begin{enumerate}
    \item Epifauna: Animals that live on the surface of the sediment
    \item Infauna: Animals that burrow in the sediment
    \item Meiofauna: Animals that live in spaces between sediment particles
    \item Including coral reefs, seaagrass beds, and kelp forests 
    \item Including the world's most important fishing ground 
    \item oil and minerals are ofund on the shelf 
    \item Close to shore, particularly vunerable to human impacts 
    \item Its relaticely shallow water and its close proximity to land 
    \item Temperature varies more
    \item Its bottom is much more affected by waves and currents than in deep water 
    \item More stable environment than intertidal 
    \item Dominated by infauna with some epidauna
    \item Two types of subtital area:
    \item [$\bullet$]Soft-bottom subtidal
    \item [$\rightarrow$]Detritus is an important food source, they come from currents, feves, dead individuals, and generated by the benthos
    \item [$\rightarrow$]Suspension feeders lives in sandy bottom while deposit feeders lives in muddy sediments 
    \item [$\bullet$]Hard-bottom subtidal
\end{enumerate}


\section{The Subtidal Ecosystems: Kelp Forests and Seagrass Beds}
\begin{enumerate}
    \item Kelp are lived in hard bottom while seagrass live in soft bottom 
    \item Kelp are lived in cold temperature regions while seagrass live in temperature and tropical regions 
    \item If there are few sea urchins, there will be a healthy kelp forest
    \item Seagrass may occur at high density
    \item It helped stablized sediments 
    \item Some herbivors feed on the leaves, but most biomass is avaliable to comsumers as detritus 
\end{enumerate}


\section{The Subtidal Ecosystems: Coral Reefs}
\begin{enumerate}
    \item The largest geological features built by organisms 
    \item Most ree-building contain photosynthestic algae, Zooxanthellae 
    \item Fringing Reef $\Rightarrow$ Barrier Reef $\Rightarrow$ Atoll
\end{enumerate}


\section{The Intertidal: Hard Bottoms}
\begin{enumerate}
    \item The Intertidal Zone can be explored without any speciallized and expensive equipment
    \item The hard bottom is dominated by epifauna
    \item It procided hard substrates for organisms to attach  
    \item Major Physical Challenges:
    \item [$\bullet$]desiccation
    \item [$\bullet$]Temperature fluctuations 
    \item [$\bullet$]Salinity changes 
    \item [$\bullet$]Wave action and tides 
    \item [$\bullet$]Oxygen availability and build-up of CO$_{2}$ at low tide 
    \item Spray Zone $\Rightarrow$ High Tide $\Rightarrow$ Mid-Tide $\Rightarrow$ Low Tide 
    \item The upper limit is governed by physical factors
    \item The lower limit is governed by biological factors 
    \item Space is the key limiting factor on Rocky Shores 
\end{enumerate}


\section{The Intertidal: Soft Bottoms}
\begin{enumerate}
    \item How much water motion and source of sediment defined what kind of sediment accumulates in an area 
    \item Coarse sediments: Areas affected by waves and currents 
    \item The smaller the defiment size, the less oxygen in the water fillinf spaces 
    \item Except the top few cm is anoxic:
    \item [$\bullet$]No oxygen
    \item [$\bullet$]Contains hydrogen sulfide, which is produced by anaerobc bacteria 
    \item [$\bullet$]Hydrogen sulfide is toxic to most animals, there has relatively little animal life 
    \item Sheltered and exposed beaches have different communities 
    \item Soft bottom are unstable and shift in response to waves, tides, and currents
    \item Seagrasses are the most common large primary producers 
    \item Infauna and deposit feeders dominate with some epifauna 
    \item Fishes and birds are key predators 
    \item Difference in teh bill length of wading shorebirds allow them to feed on particular mudflat animals (bill = the mouthpart of a bird)
    \item Suspension feeders are more common in sandy shores, while deposit feeders are more abundant in muddy shores 
    \item Most of the plants in sold-bottom intertidal area are seagrasses and benthic diatoms 
\end{enumerate}


\section{Estuaries}
\begin{enumerate}
    \item Many arine species spend at least a portion of their lives in an estuary, mmostly as larvae
    \item The fluctuating environmental conditions, mainly salinitym are so extreme that few species have evolved the necessary physiological speciallizations to survive over there. And hence there are so few Estuarine species.
    \item Three types of estuarine ecosystems:
    \item [$\bullet$]Mudflats
    \item [$\bullet$]Salt Marshes
    \item [$\bullet$]Mangroves 
\end{enumerate}


\section{The Estuarine Ecosystems: Salt Marshes and Mangroves}
\begin{enumerate}
    \item Salt Marshes:
    \begin{enumerate}
        \item Grassy vegetation that lives along the shores of estuaries and sheltered coasts in temperate and subarctic regions 
        \item The area is partially flooded at high tide
        \item Sometimes they are grouped with freshwater marshes and collectively called Wetlands
        \item Is is dominated by a few hardy grasses and other salt-tolerant land plants.
        \item 
    \end{enumerate}
    \item Mangroves:
    \begin{enumerate}
        \item Trees and shrubs that live along the intertidal shores in tropical and subtropical regions 
        \item It can be found in tropical regions
        \item Dense forest formed by mangroves are mangals
        \item Prop roots help trees remain stable substrates
        \item Aerial roots improve oxygen transport to roots in black mangroces 
        \item The muddy bottom is inhabited by a range of deposit and suspension feeders
        \item Crabs are particularly common 
    \end{enumerate}
    \item Both of them have deposit feeders but the salt marshes have filter feeders while the mangroves have suspension feeders.
\end{enumerate}


\section{Coastal Eutrophication}
\begin{enumerate}
    \item Effects of Coastal Eutrophication:
    \begin{enumerate}
        \item Harmful Algal Blooms 
        \item Dead Zones (Hypoxia)
        \item Fish Kills 
    \end{enumerate}
    \item Normal conditions:
    \begin{enumerate}
        \item Healthy food web 
        \item Healthy fisheries 
        \item Oxygenarted sediment 
    \end{enumerate}
    \item Over-enriched with nutrients:
    \begin{enumerate}
        \item Phytoplankton blooms and unhealthy food web 
        \item Dead cells use up oxygen 
        \item Anoxic sediment 
    \end{enumerate}
\end{enumerate}


\section{Effects of Harmful Algal Blooms (HABs)}
\begin{enumerate}
    \item Not all red tides are harmful, plankton blooms that produce toxins is harmful 
    \item Red tides and HAB can be used interchangeably
    \item The recent increase in HAB events off the coast of China is related tot eh increase in the use of nitrogen-based fertilizer.
    \item Impact of HABS:
    \item [$\bullet$] food web interruption, oxygen depletion, loss of biodiversity, and poisoning and deaths of fish and otehr marine vertebrates.
\end{enumerate}


\section{Case Study: Noctiluca Blooms}


\section{MC Question}
\begin{enumerate}
    \item The particular characteristic most widely used in classifying intertidal communities is
    \begin{enumerate}
        \item   type of substrate. 
        \item   relative exposure to air. 
        \item   type of seaweeds. 
        \item   type of tides. 
    \end{enumerate}
    \item The main source of food in muddy bottom intertidal communities is
    \begin{enumerate}
        \item   detritus. 
        \item   seaweed. 
        \item   epifauna. 
        \item   plankton. 
    \end{enumerate}
    \item What are expected to be relatively rare on a rocky shore.
    \begin{enumerate}
        \item   Primary producers 
        \item   Filter feeders 
        \item   Deposit feeders 
        \item   Carnivores 
    \end{enumerate}
    \item Which of the following statements is TRUE about the vertical zonation on rocky shores?
    \begin{enumerate}
        \item   The UPPER limit of a certain species can be governed by space and predation.
        \item   The UPPER limit of a certain species can be governed by predation and desiccation.
        \item   The LOWER limit of a certain species can be governed by temperature changes and wave actions.
        \item   The UPPER limit of a certain species can be governed by space and desiccation.
    \end{enumerate}
    \item One of these organisms is typically a very rare component of soft-bottom intertidal communities:
    \begin{enumerate}
        \item   Infauna. 
        \item   Deposit feeders. 
        \item   Seaweeds. 
        \item   Detritus feeders. 
    \end{enumerate}
    \item In a soft-bottomed intertidal community, oxygen would be most plentiful for meiofauna in
    \begin{enumerate}
        \item   sand. 
        \item   silt. 
        \item   mud. 
        \item   gravel. 
    \end{enumerate}
    \item The number of species inhabiting estuaries is xxx those inhabiting nearby marine or freshwater habitats mainly because of xxx.
    \begin{enumerate}
        \item   significantly higher than, temperature fluctuations 
        \item   significantly lower than, reduced predation pressure 
        \item   significantly lower than, salinity fluctuations 
        \item   similar to, similar light intensity 
    \end{enumerate}
    \item Salt Marshes and Mangroves are/have
    \begin{enumerate}
        \item   same substrate types. 
        \item   high in biodiversity. 
        \item   all of the other answers are correct. 
        \item   geographic separation in distribution (with a few exceptions). 
    \end{enumerate}
    \item Coastal eutrophication can result in all of the following, EXCEPT
    \begin{enumerate}
        \item   none of the other answers is correct. 
        \item   harmful algal bloom. 
        \item   hypoxia in deeper waters. 
        \item   economic loss. 
    \end{enumerate}
    \item The loss of estuaries and mangrove forests is particularly serious since these ecosystems
    \begin{enumerate}
        \item   provide nesting or resting areas to many seabirds. 
        \item   are among the most productive of all marine ecosystems. 
        \item   all of the other answers are correct. 
        \item   directly or indirectly provide food to many species. 
    \end{enumerate}
\end{enumerate}


\chapter{Module 3c Marine Ecosystems: The Deep Sea}
\section{The Physical Environment of the Deep Sea}
\begin{enumerate}
    \item Epipelagic Zone $\Rightarrow $ Meso Zone $\Rightarrow $ Abyssal Zone $\Rightarrow $ Hadal Zone
    \item Deep sea = Waters below the Meso Zone (1000m)
    \item Average depths of the ocean is about 4000 meters
    \item Low temperature $\Rightarrow $ Anti-freeze proteins and slow metabolic rate
    \item High pressure (Hydrostatic pressure increases by 1 atmosphere per 10m depth increase) $\Rightarrow $ Osmolytes
    \item High pressure and low temperatue affect metabolism of organisms, particularly enzyme systems, protein synthesism and physical properties.
    \item Most of the deep-sea fishes have lower metbolic rayes and slower movements
    \item No light $\Rightarrow $ Bioluminescence by symbiotic bacteria 
    \item Dilute environment $\Rightarrow $ opportunistic and energy saving adaptations
\end{enumerate}


\section{Sampling Gear for Deep-Sea Research}
\begin{enumerate}
    \item Underwater Vehicles:
    \item [$\bullet$]AUV $-$ Autonomous Underwater Vehical
    \item [$\bullet$]ROV $-$ Remotely Operated Vehicles
    \item [$\bullet$]HOV $-$ Human Operated Vehicles
    \item Constraints:
    \item [$\bullet$]Difficult and expensice 
    \item [$\bullet$]Because of low anumal densities, large smapling gear needed 
    \item [$\bullet$]The deeper the sampling depths, the greater the amount of cables required 
    \item [$\bullet$]The heaview the equipment, the bigger the ship needed 
    \item Problem in sampling:
    \item [$\bullet$]Time consuming
    \item [$\bullet$]Slow and limited catch
\end{enumerate}


\section{Challenging the Deepest Point: The Mariana Trench}
\begin{enumerate}
    \item Challenges in the Ocean Trenches:
    \item [$\bullet$]Expensive
    \item [$\bullet$]Dangerous
    \item [$\bullet$]Cannot stay on the ocean floor for long
    \item [$\bullet$]Not many smaples can be retrieved 
\end{enumerate}


\section{Discovery: Deep-Sea Chemosynthetic Ecosystems}
\begin{enumerate}
    \item Chemosynthesis is the use of inorganic compounds as energu source 
    \item Chemosynthesis is important for deep-sea organisms as total darkness in the deep-sea environment, so they must rely on chemical energy from bacteria, sucha s consume bacteria or symbiotic relationship with bacteria 
\end{enumerate}


\section{Hydrothermal Vents}
\begin{enumerate}
    \item The very hot fluid carries dissolved metals and H$_{2}$S gas from deep beneath the ocean floor 
    \item Hyftothermal vents are found along areas where tectonic plates meet 
    \item Environmental conditions around Hydrothermal Vents:
    \item [$\bullet$]High Temperature
    \item [$\bullet$]High pressure 
    \item [$\bullet$]Highconcentration of H$_{2}$S (hydrogen sulfide)
\end{enumerate}


\section{Cold Seeps and Gas Hydrates}
\begin{enumerate}
    \item Cold Seeps means chemicals poduced by the decay of organic matter seep out from the sea floor 
    \item These chemicals provide energy to support the ecosystem, such as chemosynthesis
    \item Cold seeps are found near hydrothermal vents 
    \item Near inner edge of brine pool can find lively organism, sucha s mussels, worms and crabs
    \item Outer edge of brine pool can find dead organisms becuase they are away from their energy supply
    \item Gas hydrate $-$ clean energy
    \item Worms live on hydrocarbon 
    \item Importance of Cold Seeps:
    \item [$\bullet$]Biologically: Very rich biodiversity like Hydrothermal vents 
    \item [$\bullet$]Ecologically: Very dynamic chemosynthetic ecosystem
    \item [$\bullet$]Natural Resources: Gas hydrate and hydrocarbon 
\end{enumerate}


\section{Sea Mounts, Ocean Basins and Ocean Trenches}
\begin{enumerate}
    \item Seamounts $-$ More diverse than other parts of the sea floor, a large underwater mountain that does not reach the water surface and rich in heavy metal 
    \item Seafloor $-$ Unique ecosystems and recources
    \item Ocean Trench $-$ Long, narrow depression on the ocean floor 
    \item Ocean Basin $-$ all land that is covered by sea water, including the Continental Margin
    \item Ocean Ridge $-$ A mountain-like structure formed under the sea due to colvanic eruptions ont eh ocean floor 
    \item 
\end{enumerate}


\section{MC Question}
\begin{enumerate}
    \item Bacteria thriving around deep-sea hydrothermal vents are
    \begin{enumerate}
        \item   the energy production mechanism of these bacteria remains largely unknown.
        \item   photosynthetic only. 
        \item   photosynthetic and chemosynthetic. 
        \item   mainly chemosynthetic. 
    \end{enumerate}
    \item Which of the following statements is true?
    \begin{enumerate}
        \item   Deep-sea area is much larger in size compared to that of the habitable land on Earth.
        \item   Deep-sea area is much smaller in size compared to that of the habitable land on Earth.
        \item   Deep-sea area is of similar size as that of the habitable land on Earth. 
        \item   There has been limited study in this aspect and it remains largely unknown about the comparison of the size of deep-sea area to that of the habitable land on Earth.
    \end{enumerate}
    \item Osmolytes are important for the deep-sea amphipods to adapt to the environment because they can
    \begin{enumerate}
        \item   stabilize protein structure and function. 
        \item   reduce metabolic rate. 
        \item   none of the other answers is correct. 
        \item   lower body temperature. 
    \end{enumerate}
    \item The deepest of ocean waters is classified as the
    \begin{enumerate}
        \item   Meso Zone. 
        \item   Abyssal Zone. 
        \item   Pelagic Zone. 
        \item   Hadal Zone. 
    \end{enumerate}
    \item The challenges of conducting research in the Ocean Trenches include
    \begin{enumerate}
        \item   not many samples can be collected for research from each expedition. 
        \item   all of the other answers are correct. 
        \item   the expedition is very dangerous. 
        \item   the expedition is very expensive. 
    \end{enumerate}
    \item There is/are plenty of xxx in most deep-sea habitats.
    \begin{enumerate}
        \item   photosynthetic organisms 
        \item   light 
        \item   food 
        \item   none of the other answers is correct 
    \end{enumerate}
    \item Which of the following is TRUE about deep-sea vestimentiferan tubeworms?
    \begin{enumerate}
        \item   None of the other answers is correct. 
        \item   They do not rely on symbiosis to survive. 
        \item   They are unable to grow fast near hydrothermal vents or cold seeps. 
        \item   They do not have a digestive system. 
    \end{enumerate}
    \item Deep-sea organisms have special adaptations to survive through xxx pressure and xxx temperature.
    \begin{enumerate}
        \item   low, high 
        \item   high, high 
        \item   low, low 
        \item   high, low 
    \end{enumerate}
    \item Recent discoveries have shown the Challenger Expedition and other 19th-century oceanographic expeditions' assumption that the deep ocean had no xxx  was NOT true. 
    \begin{enumerate}
        \item   volcanic activity 
        \item   sediments 
        \item   biodiversity 
        \item   water movement 
    \end{enumerate}
    \item Deep-sea animals adapt to the light condition of the habitats by
    \begin{enumerate}
        \item   bioluminescence. 
        \item   less efficient muscle enzymes. 
        \item   all of the other answers are correct. 
        \item   osmolytes. 
    \end{enumerate}
\end{enumerate}


\chapter{Module 4 Humans and the Sea}
\section{Ocean and the climate}
\begin{enumerate}
    \item Angle of incoming sunlight affects the amount of solar energy.
    \item Ocean currents, Regions' temperature, Hydrological cycle, whether and formation of tropical cyclones modulate the climate.
\end{enumerate}


\section{Ocean and global warming}
\begin{enumerate}
    \item There is an acceleration of ocean warming for all depths.
    \item Impact of ocean warming:
    \item [$\bullet$]Melting ice $\Rightarrow $ sea level rise
    \item [$\bullet$]Warming ocean $\Rightarrow$ stronger storms
    \item [$\bullet$]Less ocean mixing $\Rightarrow$ less nutrients at surface $\Rightarrow$ less biological productivity and lower oxygen level 
    \item Greenhouse effect:
    \item [$\bullet$]The trapping of the sun's heat in the lower atmosphere
\end{enumerate}


\section{Ocean acidification}
\begin{enumerate}
    \item Ocean acidifiction is the ongoing decrease in the pH of the Oeans.
    \item There is a clear connection between rising atmospheric CO$_{2}$ levels and devlining ocean pH
    \item About 25\% of human produced CO$_{2}$ dissolves in the seawater
    \item Imapct of ocean acidification:
    \item [$\bullet$]make it harder for shelled organisms to build and maintain their shells and result in shell damage (degraded)
    \item [$\bullet$]lower diversity and reduce coral
    \item [$\bullet$]Food security
    \item [$\bullet$]Coastal protection
    \item [$\bullet$]Tourism
    \item [$\bullet$]Carbon storage and climate regulation
    \item Solutions:
    \item [$\bullet$]Be mindful of carbon footprint
\end{enumerate}


\section{The Global Coral Reef Crisis}
\begin{enumerate}
    \item Coral reefs are found in shallow and warm waters
    \item Built by corals, clams, and other calcifiers
    \item Home for fish, invertebrates and larvae
    \item Corals are made up of polyps that house symbiotic Photosynthesis algae, Zooxanthellae
    \item The algae produce oxygen and provide nutrients to the coral
    \item Values of coral reef:
    \item [$\bullet$]Coral reed tourism (income)
    \item [$\bullet$]Coral reef fishing (food)
    \item [$\bullet$]Natural barriers
    \item [$\bullet$]Source of the anticancer drug (medical drugs)
    \item [$\bullet$]Ecosystem
    \item Coral bleaching:
    \item [$\bullet$]A widespread stress response to warming temperature
    \item [$\bullet$]When water is too warm, corals will expel the algae $\Rightarrow$ Carals turn completely white
    \item Coral Reefs can't withstand extended hot periods
    \item Coral Reefs become barren and skeletal
    \item Solutions:
    \item [$\bullet$]Using Oxybenzone-free sunscreen
\end{enumerate}


\section{Resources from the Ocean}
\begin{enumerate}
    \item $17\%$ of the world's animal protein is provided by ocean fish
    \item Ocean Fish $-$ Nutrient-Rich Source:
    \item [$\bullet$]Good calories and protein
    \item [$\bullet$]Good fat and cholesterol
    \item [$\bullet$]Good vitamins and minerals
    \item Aquaculture production is expanding and it must continue to expand capacity
    \item Wild fisheries is reaching the maximum capacity
    \item Chinese seafood consumption is higher than worldwide average
    \item Industrial fishing operations are often not sustainable
    \item Overfishing means removing a species of fish from a body of water at a rate that the species cannot replenish in time
    \item Overfishing will cause disruption of the marine food chain, such as fishes disappear, resulting in the disappearance of top predators and reducing bidiversity
\end{enumerate}


\section{Coastal pollution and other anthropogenic impacts}
\begin{enumerate}
    \item Chemical Pollution:
    \item [$\bullet$]Toxic Metals/ Organics
    \item [$\bullet$]Pesticides/ Herbicides
    \item [$\bullet$]Antibiotics
    \item [$\bullet$]Radioonuclides
    \item [$\bullet$]Radiation leak in Japan in 2011 increased radionuclides in marine life.
    \item Physical Pollution:
    \item [$\bullet$]Wind and currents cause greater connections to build up, such as the Great Pacific Garbage Patch
    \item [$\bullet$]Microplastics: It flow from water drains, absorb hormone disruptive chemicals. Marine organisms ingest microplastics/ microbeads and they move up the food chain.
    \item [$\bullet$]Dredging: Removing unwanted silt/ mud from water to clear water pathways or for land reclamtion. It may harm habitats and turbid water can clog fish gills.
    \item [$\bullet$]Effect of Thermal Pollution:
    \begin{enumerate}
        \item Unable to adapt and die
        \item Increase bateria levels 
        \item Reduce biodiversity
        \item Disturb the food chain 
    \end{enumerate}
    \item [$\bullet$]Shipping:
    \begin{enumerate}
        \item Cause stress and hence weakens immune system
        \item Impairs hearing \& communication
        \item Migration
    \end{enumerate}
    \item [$\bullet$]Land Revlamation
    \item [$\bullet$]Non-natice Species $\Rightarrow$ Invasion Effect:
\end{enumerate}


\section{Oyster Farming and Seafood Safety}
\begin{enumerate}
    \item Mercury Exposure:
    \begin{enumerate}
        \item Lack of coordination
        \item Muscle weakness
        \item Nerve loss in hands and face
        \item Vision changes
    \end{enumerate}
    \item Cadmium Exposure (can be seen in Oyster):
    \begin{enumerate}
        \item Kidney or lung failure
        \item Bone disease
    \end{enumerate}
\end{enumerate}


\section{Marine Protected Areas}
\begin{enumerate}
    \item Geographically defined areas that are designated for consercation to protext marine resources, including intertidal, sub-tidal, and pelagic encironments.
    \item Aim at maintaing biodiversity
    \item Protect critical habitats from damage and allow them to recover
    \item Provide areas where fish can reproduce, spawn and grow to their adult size
    \item Maintain local cultures, economies, and livelihoods.
    \item Benefits of MPAs:
    \begin{enumerate}
        \item The fish biomass increased by 4 to 5 times
        \item Length increased by 25$\%$
        \item Density of fish increased
        \item Number of species increased
        \item Coral cover increased
    \end{enumerate}
    \item Factor to be considered for No-taking Zone:
    \begin{enumerate}
        \item Overfishing location
        \item Size depending on the behavior of the species
        \item It should be a self-sustaining areas
        \item Cost of fishermen
        \item Monitoring costs, bological assessment costs and enforcement costs
    \end{enumerate}
    \item Challenges for Hong Kong marine area:
    \begin{enumerate}
        \item Unregulated fishing
        \item Marine traffic
        \item Marine pollution
        \item Reclamtion
    \end{enumerate}
    \item Experts required:
    \begin{enumerate}
        \item Scientists
        \item Consercationists
        \item Fishermen
        \item NGOs
    \end{enumerate}
\end{enumerate}


\section{MC Question}
\begin{enumerate}
    \item Which of the following statements is TRUE about microplastics?
    \begin{enumerate}
        \item   They can absorb toxic chemicals 
        \item   They can be found in the digestive tracts of marine fishes 
        \item   All of the other answers are correct. 
        \item   They can enter the marine ecosystem from water drainage. 
    \end{enumerate}
    \item Marine pollution can be caused by all of the following human activities, EXCEPT
    \begin{enumerate}
        \item   agriculture. 
        \item   none of the other answers is correct. 
        \item   aquaculture. 
        \item   the use of health and beauty products. 
    \end{enumerate}
    \item Which of the following statements is NOT true?
    \begin{enumerate}
        \item   Pollution is found in polar regions. 
        \item   None of the other answers is correct. 
        \item   The deep sea is free from pollution. 
        \item   Marine pollution is generally coupled with economic activity. 
    \end{enumerate}
    \item Which of the following statements is TRUE about ocean warming?
    \begin{enumerate}
        \item   More nutrients are available at the sea surface. 
        \item   None of the other answers is correct. 
        \item   It occurs at the sea surface only. 
        \item   It results in stronger storms. 
    \end{enumerate}
    \item Greenhouse effect is related to all of the following, EXCEPT
    \begin{enumerate}
        \item   higher oxygen level in the ocean. 
        \item   less ocean mixing. 
        \item   sea-level rise. 
        \item   ocean warming. 
    \end{enumerate}
    \item The stakeholders of Marine Protected Areas (MPAs) include all of the following, EXCEPT
    \begin{enumerate}
        \item   fishermen. 
        \item   conservationists. 
        \item   none of the other answers is correct. 
        \item   researchers. 
    \end{enumerate}
    \item Which of the following statements is True?
    \begin{enumerate}
        \item   The global aquaculture production has been decreasing. 
        \item   In terms of global seafood production, aquaculture is insignificant compared to the wild catch.
        \item   In general, industrial fishing operations are unsustainable. 
        \item   The global wild catch of seafood is at its minimum. 
    \end{enumerate}
    \item Which of the following is/are related to the sustainable future of the ocean?
    \begin{enumerate}
        \item   Natural resource management. 
        \item   Consumers’ choice and natural resource management. 
        \item   Consumers’ choice and culture. 
        \item   Natural resource management, consumers’ choice, and culture. 
    \end{enumerate}
    \item A warmer world is LEAST likely to result in
    \begin{enumerate}
        \item   increased coral cover. 
        \item   reduction in biodiversity. 
        \item   decreased food production. 
        \item   increased bacteria levels. 
    \end{enumerate}
    \item Which of the following is NOT an impact of ocean acidification?
    \begin{enumerate}
        \item   Loss of biodiversity. 
        \item   Climate regulation. 
        \item   Economic loss. 
        \item   Increased food security. 
    \end{enumerate}
\end{enumerate}



\end{document}