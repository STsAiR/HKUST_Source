\documentclass[11pt]{article}
\marginparwidth0pt
\oddsidemargin -1.4 truecm
\evensidemargin0pt
\marginparsep0pt
\topmargin -2.5truecm
\textheight 26.3truecm
\textwidth 18.9 truecm
\linespread{1.3}
\usepackage{amsthm,amssymb,amsfonts,amsmath,amstext,xcolor}
\usepackage[utf8]{inputenc}

\newtheorem{theorem}{Theorem}[section]
\newtheorem{lemma}[theorem]{Lemma}
\newtheorem{corollary}[theorem]{Corollary}
\newtheorem{proposition}[theorem]{Proposition}
\newenvironment{remark}{\noindent{\bf Remark}}{\vspace{3mm}}
\newenvironment{remarks}{\noindent{\bf Remarks }}{\vspace{0mm}}
\newenvironment{notes}{\noindent{\bf Notes}}{\vspace{0mm}}
\newenvironment{note}{\noindent{\bf Note}}{\vspace{0mm}}
\newenvironment{solution}{\noindent{\it Solution\/}}{\vspace{2mm}}
\newenvironment{question}{\noindent{\bf Question}}{\vspace{0mm}}
\newenvironment{questions}{\noindent{\bf Questions}}{\vspace{0mm}}
\newenvironment{example}{\noindent{\bf Example }}{\vspace{0mm}}
\newenvironment{examples}{\noindent{\bf Examples }}{\vspace{0mm}}
\newcommand{\QED}{\fbox{}}
\newcommand{\ds}{\displaystyle}
\newcommand{\bs}{\boldsymbol}

\begin{document}
\begin{center}
    \vspace{0.6cm}
    {\Large \bf Koo BB's Lang HW (So Lazy)}
    \vspace{0.3cm}
\end{center}


\section{Successful and Sustainable Urban Community}
Urban and sustainable:

An urban setting can be defined broadly on the basis of population density, concentration of administrative bodies and infrastructure and a diverse set of livelihood and income generation activities. Urban areas will be characterized by high population density when compared to other areas. While some cities are defined by municipal boundaries, many urban centers have not been designated as such. They are usually characterized by the presence of administrative structures such as government offices and courts and a relative concentration of services such as hospitals and financial institutions such as banks. In an urban setting, the forms of livelihood and income generation activities will be diverse and unlike rural areas not bound mainly to agricultural production. If the area in question fits some if not all of these basic characteristics, it can be regarded as urban.

According to UN World Urbanization Prospects, by 2050, with the urban population doubling its current size, nearly 70 out 100 people in the world will live in cities.



\section{Factors}
\begin{enumerate}
    \item Resources\\
    $\Longrightarrow$ According to the Planning Department of Hong Kong, around $70\%$ of land are rural area in Hong Kong.
    \item Strong bonding 

    \item Environmental

    \item Economic 
\end{enumerate}


\section{Initiatives}

\raggedright Sustainable:
\begin{enumerate}
    \item Make it easy to get around without a car\\
        $\Longrightarrow$ Decrease in congession, human emission. Improve air quality. Can be achieved by building cycle superhighways.
    \item Add electric vehicles charging stations\\
        $\Longrightarrow$ According to the research office in the Lagislative Council, there are only 18361 registered EV in Hong Kong in 2020, it occupied only $2\%$ of vehicles. 
    \item Provide access to public resources and green spaces
    \item Improve water conservation and wastewater management
    \item Support urban farming
    \item Implement green architecture
\end{enumerate}

Other Actions:
\begin{enumerate}
\item Policy Address: planning ahead 
\item Waste Blueprint for Hong Kong 2035 
\item Energy Saving Plan
\item Hk smart city blue paint
\end{enumerate}


\section{Examples}
\begin{enumerate}
    \item San Francisco:\\
    It is the fourth most populous city of California, has mandated that garages and parking lots install EV charging stations for over $10\%$ of their spaces. The city aims to achieve $100\%$ emission-free ground transportation within 20 years.
    \item Paris:\\
    France is a leader in green building tech, implementing practices such as using high-density materials that capture and release solar heat to aid in heating and cooling.
    \item Dubai:\\
    It is an entirely self-sustainable city via renewable energy. The city is powered by clean energy produced by recycling water and waste, has $60\%$ green space irrigated with gray wastewater, and has banned single-use plastic bags.
    \item New York:\\
    It is one of the most advanced cities for sustainability. The Big Apple is implementing multiple sustainability programs, from a Carbon Challenge which aims for $50\%$ emissions reduction by 2025 to a Zero Waste project, city bike rentals and urban parks built on landfill sites.

\end{enumerate}


\section{The Most Desirable and Feasible Initiatives}

\newpage

\section{Skills}
\raggedright\ Questions that encourage people to give more evidence

\begin{enumerate}
\item How do you know that?
\item Have you read anything that supports your point?
\end{enumerate}


Questions that help people to see and explain connections between ideas

\begin{enumerate}
\item Is there a connection between that and our previous idea?
\item Does your idea link with what Peter said earlier? 
\end{enumerate}

Questions that encourage people to consider other possibilities/perspectives

\begin{enumerate}
\item Is there another way of looking at this?
\item Do you think we are covering all the angles?
\end{enumerate}

\section{Original Questions}
\begin{enumerate}
    \item Which factors for successful, sustainable urban communities are best demonstrated in HK?
    \item What initiatives in HK help contribute to a successful, sustainable urban community?
    \item Which community initiatives around the world do you think are the best examples of initiatives that help create successful, sustainable urban communities?
    \item Which of these initiatives would be most desirable and feasible in HK?
    
\end{enumerate}

\end{document}
