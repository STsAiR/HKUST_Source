\documentclass[11pt]{article}
\marginparwidth0pt
\oddsidemargin -1.4 truecm
\evensidemargin0pt
\marginparsep0pt
\topmargin -2.5truecm
\textheight 26.3truecm
\textwidth 18.9 truecm
\linespread{1.3}
\usepackage{amsthm,amssymb,amsfonts,amsmath,amstext,xcolor, stix}
\newtheorem{theorem}{Theorem}[section]
\newtheorem{lemma}[theorem]{Lemma}
\newtheorem{corollary}[theorem]{Corollary}
\newtheorem{proposition}[theorem]{Proposition}
\newenvironment{remark}{\noindent{\bf Remark}}{\vspace{3mm}}
\newenvironment{remarks}{\noindent{\bf Remarks }}{\vspace{0mm}}
\newenvironment{notes}{\noindent{\bf Notes}}{\vspace{0mm}}
\newenvironment{note}{\noindent{\bf Note}}{\vspace{0mm}}
\newenvironment{solution}{\noindent{\it Solution\/}}{\vspace{2mm}}
\newenvironment{question}{\noindent{\bf Question}}{\vspace{0mm}}
\newenvironment{questions}{\noindent{\bf Questions}}{\vspace{0mm}}
\newenvironment{example}{\noindent{\bf Example }}{\vspace{0mm}}
\newenvironment{examples}{\noindent{\bf Examples }}{\vspace{0mm}}
\newcommand{\ds}{\displaystyle}
\newcommand{\bs}{\boldsymbol}

\begin{document}

\begin{center}
\vspace{0.6cm}
{\Large \bf To-Do List Year 2 Fall Semester}
\vspace{0.3cm}
\end{center}

\section{COMP2011}
\begin{enumerate}
    \item Lab 1 \textcolor{green}{(DONE)}
    \item Lab 2 \textcolor{green}{(DONE)}
    \item Lab 3 \textcolor{green}{(DONE)}
    \item Lab 4 (Before 12/10 0900) \textcolor{green}{(DONE)}
    \item Lab 5 \textcolor{green}{(DONE)}
    \item Lab 6 (Before 9/11 0900) \textcolor{green}{(DONE)}
    \item Lab 7 (Before 15/11 0900) \textcolor{green}{(DONE)}
    \item Lab 8 (Before 22/11 0900) \textcolor{green}{(DONE)}
    \item Assignment 1 (Before 22/10 2359) \textcolor{green}{(DONE)}
    \item Assignment 2 (Before 15/11 2359) \textcolor{green}{(DONE)}
    \item Assignment 3 (Before 30/11 2359) \textcolor{green}{(DONE)}
    \item Midterm (31/10 2000-2200) (Chapter 1-3) \textcolor{green}{(DONE)}
\end{enumerate}

\section{COMP2711}
\begin{enumerate}
    \item Homework 1 \textcolor{green}{(DONE)}
    \item Homework 2 (Before 9/10 2359) \textcolor{green}{(DONE)}
    \item Homework 3 (Before 28/10 2359) \textcolor{green}{(DONE)}
    \item Homework 4 (Before 13/11 2359) \textcolor{green}{(DONE)}
    \item Homework 5 (Before 30/11 2359) \textcolor{green}{(DONE)}
    \item Homework 6 (Before 30/11 2359) \textcolor{green}{(DONE)}
    \item Midterm (29/10 1400-1600) (until random variable) at LTB \textcolor{green}{(DONE)}
    \item Cheat Sheet
    \item Number Theory
\end{enumerate}


\section{LANG2030}
\begin{enumerate}
    \item Analyze Report (Before 7/10 1700) \textcolor{green}{(DONE)}
    \item Group Project Draft (Before 12/10 1500) \textcolor{green}{(DONE)}
    \item Presentation 1 (19/10) \textcolor{green}{(DONE)}
    \item Proposal Report (Before 18/11 1700) \textcolor{green}{(DONE)}
\end{enumerate}
Literature review: 
\begin{enumerate}
    \item Neutral to positive evaluation of the approach
    \item critical evaluation of the approach
\end{enumerate}




\section{MATH2001}
\begin{enumerate}
    \item Homework 1 \textcolor{green}{(DONE)}
    \item Homework 2 \textcolor{green}{(DONE)}
    \item Homework 3 (Before 7/10 2359) \textcolor{green}{(DONE)}
    \item Homework 4 (Before 14/10 2359) \textcolor{green}{(DONE)}
    \item Homework 5 (Before 25/10 2359) \textcolor{green}{(DONE)}
    \item Homework 6 (Before 31/10 2359) \textcolor{green}{(DONE)}
    \item Homework 7 (Before 2/11 2359) \textcolor{green}{(DONE)}
    \item Homework 8 (Before 22/11 2359) \textcolor{green}{(DONE)}
    \item Homework 9 (Before 30/11 2359) \textcolor{green}{(DONE)}
    \item Midterm (26/10 1900-2200) (Chapter 1-2) at LTD \textcolor{green}{(DONE)}
\end{enumerate}


\section{MATH2111}
\begin{enumerate}
    \item Homework 1 \textcolor{green}{(DONE)}
    \item Homework 2 \textcolor{green}{(DONE)}
    \item Homework 3 \textcolor{green}{(DONE)}
    \item Homework 4 \textcolor{green}{(DONE)}
    \item Homework 5 (Open at 12/10) \textcolor{green}{(DONE)}
    \item Homework 6 (Open at 21/10) \textcolor{green}{(DONE)}
    \item Homework 7 (Before at 8/11) \textcolor{green}{(DONE)}
    \item Homework 8 (Open at 7/11) \textcolor{green}{(DONE)}
    \item Homework 9 (Open at 15/11) \textcolor{green}{(DONE)}
    \item Homework 10 (Open at 26/11) \textcolor{green}{(DONE)}
    \item Midterm (15/10 1000-1130) (Chapter 1-3) (Make notes) \textcolor{green}{(DONE)}
\end{enumerate}



\section{COMP4900}
\begin{enumerate}
    \item Reflection (Before at 4/11 2359) \textcolor{green}{(DONE)}
    \item CV
\end{enumerate}



\section{Robotics Team}
\begin{enumerate}
    \item Robotics Team Tutorial Schematic \textcolor{green}{(DONE)}
    \item Robotics Team Tutorial PCB (Before 14/10 1900) \textcolor{green}{(DONE)}
    \item Robotics Team tutorial 3 (21/10 1900-2100) \textcolor{green}{(DONE)}
    \item Robotics Team tutorial 4 (27/10 1900-2100) \textcolor{green}{(DONE)}
    \item Discussion (08/11 1900) \textcolor{green}{(DONE)}
\end{enumerate}



\section{JOBS}
\begin{enumerate}
    \item Candy Service
    \item Gremod Company
    \item jason front end
\end{enumerate}


\section{2711}
In the algorithm, we are going to find the duplicate integers in a sequence of integers$(a_1, a_2, \cdots , a_n)$. The running time is $O(n^2)$ since the loop inner loop and outer loop has at most $n$ iterations. The worst-case is that every integer in the sequence is different, which means $a_1 \neq a_2 \neq \cdots \neq a_n$. In this case, the number of iterations will be $1+2+ \cdots + n-1 = \dfrac{(n-1+1)(n-1)}{2} = \dfrac{n^2-n}{2}$. Therefore, the running time for worst-case is $\Omega(n^2)$ and hence the running time is $\Theta(n^2)$.
\newpage


(d) The only case when a complete graph $K_n$ is bipartite is $n = 2$. Since when $n = 2$, the vertices can be separate into two sets $\{V_1\}$ and $\{V_2\}$. However, for $n>2$ we cannot separate into two disjoint sets since all the vertices are connected.


(e) The condition for a cycle $C_n$ can be bipartite is n is even and larger or equal to 4. Since for $V = \{v_1, v_2, \cdots, v_n\}$, there is exactly one edge between $v_i$ and $v_{i+1}$ for all $1\leq i \leq n$. Therefore, we can divide the vertices into two set with respect to even and odd number $i$, such as $\{v_1, v_3, \cdots, v_{n-1}\}$ and $\{v_2, v_4, \cdots, v_n\}$. If $n$ is odd and there is exactly one edge from $v_n$ to $v_1$, we cannot find two disjoint sets for a bipartite graph.

(f) The only condition for a complete bipartite graph have an Euler path but not an Euler circuit is \{$m$ is odd integer and $n =2$\} or \{$n$ is odd integer and $m =2$\}. If $m, n$ are both even number, it is an Euler circuit since all vertices have even degree. If $m, n$ are both odd number, the graph is neither Euler circuit nor Euler path. If $m,n$ is a pair of an odd number and an even number which is not equal to 2, there will be there will be an even number of vertices with odd degree, which violated the definition of Euler path. If we need exactly two vertices with odd degree, the only solution is one of $m, n$ is odd, while the remained one equal to 2.


\newpage
First we prove "if a directed multigraph having no isolated vertices has an Euler circuit, then the graph is weakly connected and the in-degree and out-degree of each vertex are equal." Let graph K have an Euler circuit, which means there exist a path from every vertex to every other vertex and hence K is strongly connected. If G is strongly connected, it is also weakly connected. Besides, since K has Euler circuit, it will pass through each vertex, everything the circuit passes through the vertex, it creates one in-degree and out-degree. And hence the in- and out- degree will always be equal.



Then we prove "if the graph is weakly connected and the in-degree and out-degree of each vertex are equal, then a directed multigraph having no isolated vertices has an Euler circuit." First we assume that for every vertex $v \in K$ has the same in- and out-degree. To prove the statement, we first prove that for every vertex in $K$ there exist a cycle containing it. Then we choose a vertex $a$ and the edge pointing out $(a, b)$. Since we assume the in- and out-degree are the same, we can choose any outgoing edge from $b$ and continue this process.
At each vertex other than $a$, there must be an unvisited outgoing edge. The only vertex for which there may not be an unvisited outgoing edge is $a$. Since there is always an outgoing edge we can visit any vertex other than $a$. Therefore the cycle must end at $a$. And we proved that for every vertex in $K$ there exist a cycle containing it.

After that, since we have already proved every vertex in $K$ there exist a cycle containing it. We now try to connect the circle with edges. We may choose a vertex $x$ in a circle $G_1$ and vertex $y$ in circle $G_2$, we need two edges to connect the vertex since the in- and out-degree must be the same. There for another small circle is created and the degree of vertex must be even. Therefore $K$ is a Euler circuit. 



\end{document} 